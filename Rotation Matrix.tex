\documentclass[12pt]{article}
\usepackage{amsmath, amssymb, mathtools}
\usepackage{tikz}
\usepackage{geometry}
\geometry{margin=1in}
\title{Learning Notes: Affine Transformations in 2D Geometry}
\author{}
\date{}

\begin{document}
\maketitle

\section*{Motivation}
In the problem, we are given $N$ points on a 2D plane and a series of $M$ operations such as:
\begin{itemize}
    \item Rotating all points $90^\circ$ clockwise or counterclockwise around the origin.
    \item Reflecting all points across vertical or horizontal lines.
\end{itemize}
After these operations, we are to answer $Q$ queries of the form: “What are the coordinates of point $b$ after the $a$-th operation?”

\section*{Step 1: Using $2 \times 2$ Rotation Matrices}
The standard rotation matrix for rotating a point $(x, y)$ by $\theta$ around the origin is:

\[
R(\theta) =
\begin{bmatrix}
\cos\theta & -\sin\theta \\
\sin\theta & \cos\theta
\end{bmatrix}
\]

Examples:
\begin{itemize}
    \item $90^\circ$ counterclockwise ($\theta = 90^\circ$):
    \[
    R(90^\circ) = 
    \begin{bmatrix}
    0 & -1 \\
    1 & 0
    \end{bmatrix}
    \Rightarrow (x, y) \mapsto (-y, x)
    \]
    
    \item $90^\circ$ clockwise ($\theta = -90^\circ$):
    \[
    R(-90^\circ) = 
    \begin{bmatrix}
    0 & 1 \\
    -1 & 0
    \end{bmatrix}
    \Rightarrow (x, y) \mapsto (y, -x)
    \]
\end{itemize}

\textbf{Limitation:} These $2 \times 2$ matrices can perform rotation and scaling, but not \emph{translation} or \emph{reflection over arbitrary lines}.

\section*{Step 2: Homogeneous Coordinates and $3 \times 3$ Matrices}
To perform translations and more general affine transformations, we use \textbf{homogeneous coordinates}. A point $(x, y)$ becomes:

\[
\begin{bmatrix}
x \\
y \\
1
\end{bmatrix}
\]

Now every transformation is represented as a $3 \times 3$ matrix:

\[
T =
\begin{bmatrix}
a_{11} & a_{12} & a_{13} \\
a_{21} & a_{22} & a_{23} \\
0 & 0 & 1
\end{bmatrix}
\]

\subsection*{Examples}
\begin{itemize}
    \item \textbf{Rotate $90^\circ$ clockwise:}
    \[
    \begin{bmatrix}
    0 & 1 & 0 \\
    -1 & 0 & 0 \\
    0 & 0 & 1
    \end{bmatrix}
    \]

    \item \textbf{Rotate $90^\circ$ counterclockwise:}
    \[
    \begin{bmatrix}
    0 & -1 & 0 \\
    1 & 0 & 0 \\
    0 & 0 & 1
    \end{bmatrix}
    \]

    \item \textbf{Reflect across $x = p$:}
    \[
    \begin{bmatrix}
    -1 & 0 & 2p \\
    0 & 1 & 0 \\
    0 & 0 & 1
    \end{bmatrix}
    \Rightarrow (x, y) \mapsto (2p - x, y)
    \]

    \item \textbf{Reflect across $y = p$:}
    \[
    \begin{bmatrix}
    1 & 0 & 0 \\
    0 & -1 & 2p \\
    0 & 0 & 1
    \end{bmatrix}
    \Rightarrow (x, y) \mapsto (x, 2p - y)
    \]
\end{itemize}

\section*{Step 3: Composing Transformations}
Transformations can be composed by multiplying their matrices:

\[
T_{\text{total}} = T_1 \cdot T_2 \cdot T_3 \cdots T_k
\]

Important: When applying transformations in order, the \emph{new operation matrix must appear on the left} in multiplication.

\textbf{Example:} To apply operation 1, then 2, then 3:
\[
T = T_3 \cdot T_2 \cdot T_1
\]

\section*{Step 4: Applying to a Point}
Given a point $(x, y)$ in homogeneous coordinates:
\[
\begin{bmatrix}
x' \\
y' \\
1
\end{bmatrix}
=
T \cdot 
\begin{bmatrix}
x \\
y \\
1
\end{bmatrix}
\Rightarrow
x' = a_{11}x + a_{12}y + a_{13}, \quad
y' = a_{21}x + a_{22}y + a_{23}
\]

\section*{Why This Works Well}
Instead of recomputing each point's position after every query, we precompute $T_0, T_1, \ldots, T_M$ where $T_i$ is the total transformation after $i$ operations. Then we can answer each query in $O(1)$ time.

\section*{Takeaways}
\begin{itemize}
    \item $2 \times 2$ matrices are limited to linear transformations (rotation, scaling).
    \item $3 \times 3$ homogeneous matrices can handle translation and reflection as well.
    \item Precomputing transformation matrices makes queries efficient.
    \item Order of matrix multiplication matters: compose from last to first.
\end{itemize}

\end{document}